\chapter{Raspberry Pi-based Solution}
    I have decided to create a Fedorator utilizing the Raspberry Pi computer at its core.  This design is based on the previously introduced Concept B (fig. \ref{pic:concept-b}).
    \section{Specification}
%        \todo{What, exactly, will the device consist of?}
        The device consists of:
        \begin{itemize}
            \item Raspberry Pi~3 Model~B.  Compatibility with Raspberry Pi~2 Model~B may be possible, but not a primary goal.
            \item A compatible 3.5~inch touchscreen (minimum resolution $480\times320$ pixels) positioned vertically.
            \item A microSD card holding the system and several images.
            \item Two male-to-female USB cables.
            \item A 3D printed case.
        \end{itemize}
    
    \section{Software}
        The software running on the Fedorator was written to support support both Raspbian, the Linux distribution designed and officially supported for Raspberry Pi \cite{raspbian}, and the eponymous Fedora operating system.  To faciliate easier testing, it's also possible to run the software on a regular Linux desktop.
        
        The current version of the Fedorator software may be found in the appendix.  It is also made available online on GitHub, a code hosting service, with the GNU General Public License v2.0 \cite{fedorator-github}.  On GitHub, continued development will take place.
        
        The software running on the Fedorator is written in Python and uses the free open source Kivy framework \cite{kivy} to provide a touchscreen interface.  It is is designed to start immediately after boot and stay running indefinitely.
        
        \subsection{Behavior}
            In standby, the screen shows prominently the Fedora logo as well as the name of the device (fig. \ref{pic:fedorator-screenshot-standby.png}).  It prompts the user to insert a flash drive.
            
            \screenshotfigure{fedorator-screenshot-standby.png}
                {Fedorator: the standby screen}
            
            When the user inserts a flash drive, the text changes, stating ``Tap to begin".  The size of the flash drive is also displayed at the bottom part of the screen.
            
            Tapping will transition the screen into a menu listing releases available for writing (fig. \ref{pic:fedorator-screenshot-list.png}).  This list can be browsed by a swiping a finger or stylus across the screen.  Tapping the top left corner will always bring the user back to the previous screen, as indicated by the faint ``back" text. 
            
            \screenshotfigure{fedorator-screenshot-list.png}
                {Fedorator: picking a release from the list}
            
            After tapping a desired release, a screen with details is presented (fig. \ref{pic:fedorator-screenshot-details.png}).  It contains a short official description of the release and offers to set the version and architecture, should the user wish for something other than the defaults.
            
            \screenshotfigure{fedorator-screenshot-details.png}
                {Fedorator: taking a look at the Fedora Workstation release}
            
            If the user is happy with the choice, they can press the sizeable ``Flash" button.  Before the flashing operation begins, a popup appears warning the user that all data on the flash drive will be destroyed.  The user must acknowledge this with the ``Continue" button, and that begins the process of writing the image to the flash drive.
            
            %\todo{(fedorator) Do I want to change the button from "Flash" to "Write" or such?  Also, Maybe add a warning about this being a destructive action}
            
            A progress bar shows how far along the write is.
            
            After the write is complete, the screen shows ``Finished!", pulsating in green to grab attention.  It also tells the user they may now safely remove the flash drive.  A single tap at this point will bring the Fedorator back to the standby screen.
            
            \screenshotfigure{fedorator-screenshot-done.png}
                {Fedorator: Fedora LXDE Destkop was sucessfully written to the flash drive}
            
            The device returns to standby mode after a tap or after 30 seconds.  If at any point before the write finishes the flash drive is removed, the software goes back to the standby screen immediately.
            
            %\todo{more words?}
        \section{Technical Implementation}
            The Fedorator software leverages features provided by the Linux operating system for its feature set.  Following the widely discussed UNIX ``Everything is a File" philosophy \cite{ph7spot-unix-file}, certain convenient directories provide information about the connected devices after the kernel takes care of them.
            
            In particular, \texttt{/dev/disk/by-path} is used to watch for newly connected USB storage devices, and \texttt{/proc/partitions} is queried for information on the size of the medium.
            
            The primary function is the flashing operation.  After a write is initiated, the \texttt{umount} command is called on the target partition, because writing to a device which is mounted may result in corruption.
            
             Next, the software performs a direct copy to the USB drive.  This is akin to the \texttt{dd} utility.  In order to achieve the best performance, the \texttt{sendfile()} \cite{man-sendfile} function is used.  Because \texttt{sendfile()} performs copying directly in the kernel, it is faster than an equivalent task involving reads and subsequent writes.
            
            After the copy finishes from the perspective of the system, it is actually not yet safe to remove the flash drive.  Flash drives keep an internal buffer and may need more time in order to finish writing the data.  By utilizing the \texttt{udisksctl power-off} command, we issue the system to shut down and unplug the USB device as soon as it stops being busy.
            
    \section{Case}
        \imagefigurelarge{openscad-case-screenshot.png}
            {A screenshot of the OpenSCAD software with the case open.  Both the source code and the rendered model can be seen}
        I created the model for the case using the \textbf{OpenSCAD} software.  In OpenSCAD, objects are designed using a descriptive language.  One consequence is that when a value such as the length of a certain sub-component is used in several places, it may easily be defined once and used by its name repeatedly.  This allows for easy changes to just about any relevant dimension and also enhances understanding of the source code.  In proper terms, the model I designed is \textit{parametric}.
        
        OpenSCAD allows exporting the model to the STL format.  This format can then be processed by a slicing program like Slic3r or Cura to prepare for printing.
        
        \imagefigurelarge{case-render.png}
            {A render of the constructed case, courtesy of OpenSCAD}
        
        The case I designed for the Fedorator is split into two components, which may be defined as inner and outer, or bottom and top.  These two components are printed separately, but lock together when the device is constructed.
        
        The \textbf{inner part} consists of the floor, a pair of supports designated for USB connectors, and a holder for the Raspberry Pi and display pair.  The enclosure for the Raspberry Pi was designed with meticulous attention to dimensions, otherwise the device would not fit or move too much.
        
        The \textbf{outer part} is majorly the shell and cap.  There are four holes in the shell, one for the display, two for the USB ports, and one in the back for the power cable.
        
        Both parts are designed in such a way that they fit snugly together.  The top part can be carefully put over the bottom part and holds well unless taken by force.
        
        In a misguided attempt at making the device easy to assemble, I had decided to forego any screws or fastening bolts in the design.  This means the components inside hold by gravity, or are fixed by other means such as a tighening strap.
        
        %\todo{anything else? maybe a "for example, the height can be adjusted easily..."}
        
    \section{Construction of a Prototype}
%        \todo{A prototype will be built.  This should probably include pictures and a short description of the process}
        I have put a functional prototype of the Fedorator together.
        
        The prototype consists of all the required components.  The screen is a 3.5 inch HDMI touchscreen display with support for up to $1920\times1080$ pixels in resolution \cite{aliexpress-touchscreen}.  However, in the prototype, the resolution is kept as $480\times320$ to ensure broadest support.  Sadly, drivers for this display are proprietary and only provided for Raspbian \cite{osoyoo-drivers}.  Hence, the prototype runs the Raspbian operating system.
        
        The case for the prototype was printed on a LulzBot Mini \cite{lulzbot-mini} 3D printer.  The PLA material was used for filament.
        
        \imagefigure{photos/fedorator-prototype-test-print2.jpg}
            {A test print with a Raspberry Pi seated inside}
        
        In order to test the model's dimensions and fitness, a series of test prints was done first.  After certain parameters were verified, such as whether the Raspberry Pi is seated correctly, the full case was set to be printed.  The poor behavior of the test prints with regard to bridges has led to the decision to print the full case with supports\footnote{Supports are brittle, easy to remove material printed under overhangs in the object.}.
        
        Due to the printer's dimensions, the height of this prototype had to be lowered to 152~cm and both parts was printed individually.  Each part took about ten hours to print.% \todo{exact numbers?}
        
        \imagefigure{photos/fedorator-case-supports.png}
            {The case print complete with supports}
        
        The prints were successful and the device ready to assemble.
        %\todo{Photos from assembly?}
        
        \newpage
        
        
        \idkijustwanttwotofitonapagefigure{photos-marekzehra/fedorator-ready2.jpg}
            {The Fedorator prototype with two USB flash drives plugged in}
            
        \idkijustwanttwotofitonapagefigure{photos-marekzehra/fedorator-done.jpg}
            {The Fedorator prototype done flashing the Fedora Workstation system}
        
        \newpage
        
        
    \section{Instructions}
        Instructions on how to build a Fedorator are included in the BUILD.md file of the public repository, also present in the appendix.
        
        \begin{itemize}
            \item \textbf{Building a Fedorator}: A short intro tells the reader about the Fedorator being open hardware.
            \item \textbf{Components}: The necessary components are listed, with a suggestion on where to buy them.
            \item \textbf{Software}: Contains instruction on how to set up the Fedorator software on the Raspberry Pi.
            \item \textbf{Hardware-specific setup}: Gives some tips on things that may differ between configurations.
            \item \textbf{Assembly}: Contains a description of the case, instructions on printing it and how to put it all together.
        \end{itemize}
        
    \section{Speed Measurements}
        To test performance in different situations, I have measured how long the Fedorator takes to write the Fedora Workstation image (1.34~GiB) to several different flash drives.  Several measurements were done in order to gain the average.  There was negligible deviation.
        
        \begin{table}[htbp]
        \centering
        \caption{Fedorator speed measurements}
        \label{usb-duplicators}
            \begin{tabular}{ m{8em} m{8em} r r }
            \toprule
                \textbf{Vendor} & \textbf{Name} & \textbf{Capacity} & \textbf{Average time} \\
            \toprule
            
       Sony                         & & 15.8~GiB & 6~m~07~s \\
\hline Sony   & MicroVault            &  1.9~GiB & 6~m~27~s \\
\hline LaCie  & iamaKey V2            & 15.1~GiB & 2~m~08~s \\
\hline PetalIT                      & & 14.6~GiB & 2~m~46~s \\
\hline Silicon Motion & Flash Drive   & 15.0~GiB & 4~m~26~s \\
\hline Silicon Motion & Flash Drive   &  3.7~GiB & 3~m~38~s \\
\hline Corsair & Flash Voyager        &  3.8~GiB & 4~m~34~s \\
\hline Unknown\footnotemark
                                    & &  3.8~GiB & 6~m~59~s \\
\hline
            \end{tabular}
        \end{table}
        % XXX this needs to be on the same page as the table!
        \footnotetext{\footnotemark This flash drive is inscribed with the text ``Ústav leteckého zdravotnictví Praha" and Linux identifies it only as ``Unknown counterfeit flash drive".}
        
        We can see that the time varies a lot depending on the speed of the particular flash drive.  Even flash drives from the same vendor can differ in their writing speed.
        
        \imagefigure{photos/various-flash-drives.jpg}
            {The various flash drives tested}
        
        
        \begin{comment}
        \todo{These are the final instructions that will be the "result" of the work.  Should they be inlined here?}
        \subsection{Bill of Materials}
            \todo{What is needed to get started - include a table and cost}
            \blind[3]
        \subsection{Step-by-step tutorial}
            \todo{The meat: how to build it from grounds up}
            \todo{Will not be present here, rather as an attachment.}
        \end{comment}
    \section{Conclusion}
        I have devised and designed a device fulfilling the specified requirements for a Fedorator.  A prototype was built and is functional.  It is ready to be tested in the field.
        
        Overall, creating the device was a success, although there is still work to be done.  I'm not satisfied with the components used for the prototype, in particular, the proprietary screen.  I would like to be able to recommend a standard screen, so I will be searching for alternatives.
        
        Another issue is the lack of fixation for the components.  During regular usage, the screen may wiggle a small amount or be at a slightly off angle.  The next iteration on this design will include screws, fasteing bolts, or a similar ``tried and tested" solution.
        
        Testing has shown the current version of the software works according to the specifications.  It is built with provisions for future improvements.  Continued development may bring features such as the addition of secondary data partitions to the resulting flash drives.
        
        %\todo{I still haven't done the speed measuring thing...}
