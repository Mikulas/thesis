\chapter{Implementace}

Tato příloha popisuje technické provedení dílčích částí práce. Především dokumentuje obecné procesy shodující se pro všechny porovnávané \CICD systémy. Konkrétní detaily jednotlivých systémů jsou zdokumentovány v hlavní části práce.

Self-hosted \CICD systémy jsem nasazoval v lokálním virtuálním stroji za pomoci prostředí Vagrant \cite{hashimoto-vagrant}\cite{susanka-vagrant}. Pro tuto práce jsem využil současně nejaktuálnější verzi \code{2.2.2}. Jako virtualizační jádro jsem použil Oracle VirtualBox \cite{virtualbox} ve verzi \code{5.2.22-126460-OSX}.

Jako základ každé instalace jsem vybral Ubuntu \cite{ubuntu}, která je podle W3Techs s 38,1 \% nejpoužívanější Linuxová distribuce \cite{w3techs-stats}. Zvolil jsem aktuální vydání LTS (long-term support) \code{18.04}, oficiálně publikované jako Vagrant box \code{ubuntu/bionic64}.

K nainstalování závislostí na čistý operační systém i k instalaci samotných aplikací jsem využil software Chef \cite{chef}. Jde o systém pro správu konfigurace a podporuje vývoj ve stylu \textit{Infrastructure as Code} (infrastruktura v kódu, oproti „klikacímu“ nastavování někde v \glstext{GUI}). Všechny konfigurace a nastavení systémů jsem tak mohl verzovat a sdílet na příloženém médiu. Veškeré popsané experimenty by tak měly být naprosto opakovatelné a spustitelné s minimem další práce.
