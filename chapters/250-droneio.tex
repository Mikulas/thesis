\section{Drone.io}

    divna dokumentace kde chybi jak to instalovat na bare metal
    pri pullovani docker image neni v adminu videt co se deje
    login pres oauth vybraneho providera, pravdepodobne nejde mit vic provideru najednou => ale dobra integrace
    v dokumentaci neuvadi verze ale doporucuji je je pouzivat, wtf
    minimalisticke ale dostacujici UI (skoda ze je to SPA, obcas je to zaseknute)
    chybi administrace, neni videt seznam jobu agentu atd
    pluginy = kontejnery s trochu jinym env a hezci syntaxi
    zabudovane sifrovani pro secrets (nic jineho to nema) + podpora pro vault atd (to ma i concourse treba)

    \subsection{Architektura Drone.io, možnosti konfigurace}
        \todo{Jaké má Drone.io části, jak se to deployuje, externí závislosti\ldots}\blind[4]

    \subsection{Rozšiřitelnost}
        \todo{Má Drone.io nějaké pluginy, dá se pro to scriptovat, jak je to bezpečné a jednoduché?}\blind[3]

    \subsection{Zabezpečení}
        \todo{Jaké jsou historická CVE? Jaká je izolace klientů? Co aplikace potřebuje za přístupy?}\blind[3]

    \subsection{Dostupnost}
        \todo{Může Drone.io běžet ve víc replikách? Jak se dělá upgrade? Jak stabilní to je?}\blind[3]

    \subsection{Integrace}
        \todo{Integrace Drone.io, oznámení na GitHub/GitLab/Bitbucket/\ldots}\blind[2]
        \todo{Možnosti deploy z Drone.io do cílového systému; k8s, sftp, openstack, \ldots}\blind[2]

    \subsection{Praktické nasazení projektů}
        \subsubsection{Projekt 1}
            \todo{Popsat deploy projektu 1 z Drone.io}\blind[2]

        \subsubsection{Projekt 2}
            \todo{Popsat deploy projektu 2 z Drone.io}\blind[2]

        \subsubsection{Projekt 3}
            \todo{Popsat deploy projektu 3 z Drone.io}\blind[2]
