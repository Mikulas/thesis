\section{Drone}
    Drone je minimalistické \CI postavené na kontejnerech. Je to svým způsobem jenom Docker orchestrátor, který má navíc drobné webové rozhraní. V \glstext{UI} lze dělat pouze dva úkony: číst výstup jobů a zapínat/vypínat sledování repozitářů. Kromě toho je celý web implementovaný jako \glstext{SPA}. To má sice pozitivní vliv na rychlost přepínání stránek, ale při testování občas aplikace nereagovala a několikrát se zasekla úplně. Dokonce ve webové administraci není ani seznam agentů, joby čekající na zpracování a podobně.

    Některé důležité funkce skrývá Drone za placenou licenci. Nejde ani o korporátní podporu, ale o nezbytné vlastnosti bez který lze \CI těžko provozovat. Mezi ně patří: dynamický runner, který by umožňoval využívat cluster nebo externí služby (například \glstext{AWS} CodeBuild) podle vytížení; sdílení tajemství napříč organizací a obecně podpora pro externí správce tajemství; šablony pro pipeline.

    \subsection{Architektura Drone, možnosti konfigurace}
        Dokumentace Drone je dostatečně obsáhlá, ale není vyčerpávající a odkazy jsou navíc netypicky pojmenované. Instalace je rozdělena podle externího správce úložišť, tradičně bývá odlišná dokumentace pro způsoby instalace (bare metal, kontejnery, \ldots). Není možné v jedné Drone instanci využívat zároveň například GitLab a Bitbucket, podporováno je pouze jedno úložiště. Dokumentační stránka o instalaci na Kubernetes je dokonce úplně špatně. Doporučuje spouštět Drone server jako \code{Pod} (mělo by jít o \code{Deployment}, nebo Helm chart) a agenti dokonce naprosto chybí. Další problém dokumentace je, že používá ve všech ukázkách Docker obrazy bez tagu. To je špatně, protože výchozí tag \code{latest} se může kdykoliv změnit na nekompatibilní verzi. Vždy je lepší explicitně uvést tag.

        Stejně jako ostatní \CI se Drone skládá z jednoho masteru a agentů. Master persistuje data do sqlite databáze, ale některá data o frontě požadavků udržuje pouze v paměti~\cite{drone-ha}. Nemá žádné externí závislosti.

    \subsection{Rozšiřitelnost}
        Drone rozlišuje dva koncepty. První jsou \textit{pipeline plugins}, což jsou obyčejné kontejnery spuštěné v pipeline. Jediný rozdíl je drobné syntaktické zjednodušení, které umožňuje v konfiguraci pipeline psát místo \code{environment[PLUGIN\_KEY]=x} jenom \code{settings.key=x}. Kdyby Drone tuto funkci neměl, byla by tvorba pluginů transparentnější a přístupnější i těm nejméně zkušeným uživatelům.

        Druhý koncept rozšíření -- dostupný pouze v placené Enterprise verzi -- upravuje nějakým způsobem definici pipeline. Jediné zdokumentované rozšíření je zatím Jsonnet, které umožňuje používat stejnojmenný šablonovací jazyk místo \glstext{YAML}~\cite{drone-jsonnet}.

    \subsection{Zabezpečení}
        Díky kontejnerové architektuře má Drone základní izolaci i v rámci jednoho agenta. Lepší izolace lze dosáhnout v placené verzi, která nabízí dynamické agenty, které můžou spouštět \glstext{VM} nebo nějakým způsobem využívat cloud.

        Vynikající vlastnost kterou ostatní v základu \CI nenabízejí je možnost zašifrovat tajemství a umístit ho do veřejně čitelné pipeline. Využívá se k tomu asymetrické šifrování, při kterém \CI nikdy nezveřejňuje svůj privátní klíč. Dál je podporován mj.~HashiCorp Vault a \glstext{AWS} Secrets Manager.

        Při napojení na GitHub je vyžadováno moc práv, včetně čtení a zápisu do všech soukromých repozitářů. Je na to otevřené issue od začátku roku 2018, ale Drone je zatím bez opravy~\cite{drone-github-acl}.

        Nenašel jsem žádná Drone \glstext{CVE}, ani jsem nenašel žádné zdokumentované bezpečnostní problémy ve veřejném seznamu chyb a úkolů celé Drone organizace (ať otevřené, nebo vyřešené).

    \subsection{Dostupnost}
        Podobně jako ostatní \CI má Drone jeden stavový master, který není možné load balancovat. Jeden z autorů prohlásil, že na stabilním cloudu má Drone dostupnost $99,999$~\%~\cite{drone-ha}. Není ale možné, aby zahrnoval i aktualizace. Kromě toho s nástupem smýšlení \textit{Infrastructure as Code} není běžné udržovat dlouhožijící \textit{pet} servery.

        Stejně jako Concourse persistuje i Drone data do databáze (rozdíl je v tom, že Concourse má master nestavový). Stačí tak zálohovat standardními prostředky které podporuje vybraný \glstext{RDBMS}. Podporované systémy jsou MySQL a Postgres~\cite{drone-database}.

        Dále je nutné zálohovat objektové úložiště, které může být volitelně používáno pro ukládání výstupu pipeline.

        Drone není dostatečně aktivně vyvíjený projekt a má teprve verzi 1.0. Roky existoval ve verzích 0.x a vznikl zatím nekompletní oficiální nástroj na migraci databáze z původní verze 0.8 na aktuální 1.0~\cite{drone-mig}. Vzhledem k tomu, že Drone vývojáři kromě tohoto vydání databázové schéma neupravují, je možné rychle aktualizovat na minor a patch verze a případně i dělat rollback. Nelze ale udělat rolling update a je pro aktualizaci nutné akceptovat krátký výpadek služby.

    \subsection{Integrace}
        Drone podporuje jenom vybrané správce repozitářů: GitHub, Gitlab, Bitbucket, Gitea a Gogs. Důležité je, že neexistuje podpora pro obecný git remote.

        Ani přes úzkou integraci na repozitáře není ale výsledná integrace bezchybná. Například oznámení výsledku pipeline na GitHub (tzv. \textit{Commit Status} nebo novější \textit{Checks}) nejsou zabudované a je nutné použít plugin, což dále komplikuje pipeline kterou uživatel musí napsat a udržovat.

        %\todo{Možnosti deploy z Drone do cílového systému; k8s, sftp, openstack, \ldots}\blind[1]

    \subsection{Praktické nasazení projektů}
        \subsubsection{Projekt 1}
            Pro statický projekt jsem jako u ostatních \CI vytvořil dvoukrokovou pipeline, kde první část kompiluje zdrojové soubory nástrojem Jekyll a druhá část nahrává výsledný web na webový server pomocí \code{rsync}. Protože všechny kroky běží v kontejneru a nelze použít zdroje na hostitelském stroji, místo přednahrání \glstext{RSA} klíče je nutné ho umístit do pipeline. Aby byl dostupný pouze pro \CI a nemohl ho nikdo jiný zneužít, je zašifrovaný \glstext{CLI} nástrojem \code{drone encrypt}. Tato utilita nepodporuje načítání z \code{stdin}, takže binární, víceřádkové nebo obecně složitější data je nutné nějakým způsobem překódovat. V tomto případě jsem klíč nejprve překódoval s \code{base64}. Toto řešení není ideální, protože vyžaduje práci v kontejneru, kde dekódovací nástroj nemusí být nainstalovaný.

            Další problém na který jsem při tvorbě pipeline narazil byla chybějící indikace stavu pipeline ve webovém rozhraní. Při prvním dlouhém stahování docker obrazu pro danou pipeline vypadá Drone rozbitě/zaseknutě. Chybí jakákoliv indikace postupu nebo dokonce času k dokončení.

            Přes všechny počáteční nesnáze je konfigurace Drone pipeline relativně snadná a přímočará. Především tomu pomáhá koncept \textit{Volumes}, což definuje složku souborů sdílející se mezi všemi joby. Ostatní \CI toto řeší mnohem složitěji, například přes read-only \textit{artifacts}.

        \subsubsection{Projekt 2}
            Implementoval jsem opět dvoukrokovou pipeline: kompilaci a deploy. Narazil jsem na drobný problém při přejmenování použitého volume kterým se předávají data mezi jednotlivými kroky. Drone nijak nevaruje, když název disku v kroku neodpovídá žádnému klíči name v poli \code{volumes} a pouze ho tiše vytvoří. Bylo by vhodné uživatele o chybě informovat, protože v lepším případě je definice pipeline nekonzistentní a v horším případě úplně nefunkční, což nastane když když dva kroky místo sdílení dat využívají chybně jiné disky.

        \subsubsection{Projekt 3}
            Stejně jako u Concourse je i u Drone kompilace kontejnerizované aplikace výrazně jednodušší. Využil jsem oficiální rozšíření \code{drone-docker}, které pro uživatele abstrahuje \glstext{DinD}. Bohužel je to velmi naivní implementace, díky které nefunguje Docker cache a všechny buildy jsou výrazně pomalejší, než musí být. Je ale možné místo tohoto oficiálního rozšíření použít vlastní dedikovaný Docker daemon. Toto je detailně popsáno \pfxref{v sekci}{sec:dind}.
