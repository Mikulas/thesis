\newcommand{\circleci}{\xspace{}CircleCI\xspace}
\newcommand{\travis}{\xspace{}Travis~\glstext{CI}\xspace}
\newcommand{\semaphore}{\xspace{}Semaphore~\glstext{CI}\xspace}

\section{\glstext{SaaS}: \circleci, \semaphore, \travis}
    V této sekci shrnu výhody a nevýhody moderních \glstext{SaaS} \CI. Vyzdvihnu významné rozdíly, pokud na nějaké narazím, ale primárně budu popisovat \circleci, \semaphore a \travis dohromady.

    \subsection{Architektura SaaS CI a možnosti konfigurace}
        \travis a \semaphore mají prakticky z pohledu uživatele prakticky stejnou architekturu. Pro každý job zapnou samostatný virtuální stroj. \travis poměrně překvapivě přešel v roce 2019 kompletně na virtuální stroje; dříve umožňoval spouštět i Docker kontejnery, což využívalo 45 \% všech jobů \cite{travis-arch}. \travis jako důvod uvádí složitější kompilace Docker obrazů pomocí \glstext{DinD}. To ale může znamenat, že pro firmu je to dražší řešení na podporu, ne nutně že jde o lepší řešení pro uživatele. V rámci virtuálního stroje má uživatel veškerou volnost a může instalovat a spouštět vesměs cokoliv. V základním obrazu je předinstalovaná celá řada často používaných nástrojů a runtime ve spoustě verzí. Velmi praktická funkce \travis, kterou ostatní \CI nástroje v základu nemají, je \textit{Build Matrix}: uživatel specifikuje různé verze různých závislostí a \CI pak spustí job pro \textit{všechny} kombinace \cite{travis-build-matrix}. Některé kombinace lze navíc označit jako volitelné a jejich selhání je jenom informační. To se hodí pro predběžné testování nestabilních \glstext{RC} verzí závislostí a podobně.

        \circleci naopak v roce 2018 zmigroval všechny uživatele z virtuální strojů (\circleci 1.0) na čistě kontejnerové prostředí (\circleci 2.0) \cite{circle-migration}. V specifikaci pipeline uživatel uvádí všechny kontejnery které chce spustit a jejich prolinkování/pořadí. Umí také spustit některé služby paralelně a na pozadí, což se používá například pro databáze a podobné závislosti.

    \subsection{Rozšiřitelnost}
        \todo{Má CircleCI, Semaphore, TravisCI nějaké pluginy, dá se pro to scriptovat, jak je to bezpečné a jednoduché?}\blind[2]

    \subsection{Zabezpečení}
        Ani jeden z těchto \glstext{SaaS} \CI systémů nenabízí bug bounty. Pro \travis jsem našel jednu zprávu o bezpečnostní chybě z roku 2018 \cite{travis-db-drop}. Pro \circleci ani \semaphore jsem žádné zveřejněné incidenty nenašel.

    \subsection{Dostupnost}
        \circleci, \travis ani \semaphore nedefinují žádné \glstext{SLA}. Za posledních 12 měsíců měl \circleci uptime $99.90$~\%~\cite{circle-uptime}, \travis $99.93$~\%~\cite{travis-uptime}. \semaphore reportuje za poslední rok podezřele vysoký uptime $100$~\%~\cite{semaphore-uptime}.

        Kromě samotných \CI služeb jsou ale tyto služby závislé na dostupnosti úložišť kódu (GitHub, GitLab, Bitbucket, …).

    \subsection{Integrace}
        \travis lze používat pouze s repozitáři na GitHub, \circleci a \semaphore podporují kromě toho také Bitbucket. Nelze používat vlastní repozitáře a jiné systémy. Teoreticky lze vytvořit vlastní obálku nad cizím \glstext{API} a posílat webhooky ve formátu jako GitHub, ale nejde o oficiálně podporovanou variantu.

        Všechny tři \CI podporují GitHub perfektně a i nově zveřejněné funkce implementují rychle.

    \subsection{Praktické nasazení projektů}
        \subsubsection{Projekt 1}
            \todo{Popsat deploy projektu 1 z CircleCI, Semaphore, TravisCI}\blind[2]

        \subsubsection{Projekt 2}
            \todo{Popsat deploy projektu 2 z CircleCI, Semaphore, TravisCI}\blind[2]

        \subsubsection{Projekt 3}
            \todo{Popsat deploy projektu 3 z CircleCI, Semaphore, TravisCI}\blind[2]
