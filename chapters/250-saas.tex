\newcommand{\circleci}{\xspace{}CircleCI\xspace}
\newcommand{\travis}{\xspace{}Travis~\glstext{CI}\xspace}
\newcommand{\semaphore}{\xspace{}Semaphore~\glstext{CI}\xspace}

\section{\glstext{SaaS}: \circleci, \semaphore, \travis}
    V této sekci shrnu výhody a nevýhody moderních \glstext{SaaS} \CI. Vyzdvihnu významné rozdíly, pokud na nějaké narazím, ale primárně budu popisovat \circleci, \semaphore a \travis dohromady.

    \subsection{Architektura SaaS CI a možnosti konfigurace}
        \travis a \semaphore mají prakticky z pohledu uživatele prakticky stejnou architekturu. Pro každý job zapnou samostatný virtuální stroj. \travis poměrně překvapivě přešel v roce 2019 kompletně na virtuální stroje; dříve umožňoval spouštět i Docker kontejnery, což využívalo 45 \% všech jobů \cite{travis-arch}. \travis jako důvod uvádí složitější kompilace Docker obrazů pomocí \glstext{DinD}. To ale může znamenat, že pro firmu je to dražší řešení na podporu, ne nutně že jde o lepší řešení pro uživatele. V rámci virtuálního stroje má uživatel veškerou volnost a může instalovat a spouštět vesměs cokoliv. V základním obrazu je předinstalovaná celá řada často používaných nástrojů a runtime ve spoustě verzí. Velmi praktická funkce \travis, kterou ostatní \CI nástroje v základu nemají, je \textit{Build Matrix}: uživatel specifikuje různé verze různých závislostí a \CI pak spustí job pro \textit{všechny} kombinace \cite{travis-build-matrix}. Některé kombinace lze navíc označit jako volitelné a jejich selhání je jenom informační. To se hodí pro predběžné testování nestabilních \glstext{RC} verzí závislostí a podobně.

        \circleci naopak v roce 2018 zmigroval všechny uživatele z virtuální strojů (\circleci 1.0) na čistě kontejnerové prostředí (\circleci 2.0) \cite{circle-migration}. V specifikaci pipeline uživatel uvádí všechny kontejnery které chce spustit a jejich prolinkování/pořadí. Umí také spustit některé služby paralelně a na pozadí, což se používá například pro databáze a podobné závislosti.

        \todo{[obrazek] circleci workflows atd}
        \missingfigure[figwidth=\columnwidth,figheight=7cm]{Architektura \label{pic:circle-architecture}}

        \semaphore kombinuje architekturu dvou předchozích řešení: jednotlivé joby běží ve virtuálních strojích, ale jsou provázané přes koncept bloků a data si předávají přes cache. Oproti \circleci chybí možnost zrychlit přípravu prostředí a předinstalaci závislostí Docker obrazem a přitom je na \semaphore definice pipeline výrazně složitější, než na \travis.

    \subsection{Zabezpečení}
        Ani jeden z těchto \glstext{SaaS} \CI systémů nenabízí bug bounty. Pro \travis jsem našel jednu zprávu o bezpečnostní chybě z roku 2018 \cite{travis-db-drop}. Pro \circleci ani \semaphore jsem žádné zveřejněné incidenty nenašel.

    \subsection{Dostupnost}
        \circleci, \travis ani \semaphore nedefinují žádné \glstext{SLA}. Za posledních 12 měsíců měl \circleci uptime $99.90$~\%~\cite{circle-uptime}, \travis $99.93$~\%~\cite{travis-uptime}. \semaphore reportuje za poslední rok podezřele vysoký uptime $100$~\%~\cite{semaphore-uptime}.

        Kromě samotných \CI služeb jsou ale tyto služby závislé na dostupnosti úložišť kódu (GitHub, GitLab, Bitbucket, …).

    \subsection{Rozšiřitelnost}
        Pro \glstext{SaaS} řešení nelze uvažovat systémy pluginů a rozšíření. Veškerá funkcionalita služby musí být přímo integrovaná v systému. Tyto možnosti popisuji v následující podsekci.

    \subsection{Integrace}
        \travis lze používat pouze s repozitáři na GitHub, \circleci a \semaphore podporují kromě toho také Bitbucket. Nelze používat vlastní repozitáře a jiné systémy. Teoreticky lze vytvořit vlastní obálku nad cizím \glstext{API} a posílat webhooky ve formátu jako GitHub, ale nejde o oficiálně podporovanou variantu.

        \semaphore má složité zakládání nového repozitáře. Na rozdíl od ostatních dvou \CI je nutné nainstalovat na klientu binárku, která podporuje pouze Linux a macOS a instaluje se přes \code{curl|bash}. Poté se v repozitáři musí zavolat \code{sem init}, který ale funguje pouze pokud má projekt nastavený \code{origin remote} na GitHub. Běžně stačí projekt přidat v administraci \CI: díky propojení na GitHub \CI ví, jaké projekty existují.

        Všechny tři \CI podporují GitHub perfektně a i nově zveřejněné funkce implementují rychle.

    \subsection{Praktické nasazení projektů}
        \subsubsection{Projekt 1}
            Ze všech \CI vyzkoušených v této práci bylo nasazení na \travis suverénně nejjednodušší. Na deseti řádcích se přehledně definují všechny závislosti a volá se build. I nezaškolený uživatel by dokázal vytvořit novou pipeline podle minimální ukázky.

            Na \circleci je konfigurace pipeline znatelně složitější. Rozdělením na kontejnery se ale separují jednotlivé závislosti a správa komplexních projektů by byla jednodušší. Další výhoda \circleci pro tento projekt je možnost opakovat pouze jenom dílčí kroky a ukládat mezivýsledky do cache, která se může použít při každém spuštění. Toho jsem využil pro instalaci závislostí z Gemfile. Alternativou bylo vytvořit nový Docker obraz a Ruby gemy tam předinstalovat. To má ale řadu nevýhod, předně to zvyšuje komplexitu a bylo by složitější použít jiné gemy (jiná rozšíření pro Jekyll).

            Pro \semaphore jsem de facto musel zkombinovat obě předchozí řešení: v \glstext{VM} jsem nechal nainstalovat Ruby gemy a uložil je do cache. V druhém jobu se cache stáhne a spustí se kompilace samotného statického webu.

            Na rozdíl od ostatních nasazení jsem při testování \glstext{SaaS} neimplementovat celou \CI pipeline včetně nahrání na cílový server. Všechny systémy jsem zprovoznil v lokálním virtuálním prostředí na které není vhodné dělat vzdálený přístup. Místo \code{make deploy} je tak v definicích pouze job s hláškou, kde by se deploy spouštěl.

        \subsubsection{Projekt 2}
            \todo{[PROGRAMOVANI] Popsat deploy projektu 2 z SaaS}\blind[2]

        \subsubsection{Projekt 3}
            \todo{[PROGRAMOVANI] Popsat deploy projektu 3 z SaaS}\blind[2]
