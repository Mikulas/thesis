\label{Úvod}
Proces vývoje software, při kterém změny všech vývojářů začleňujeme do společného kódu co nejrychleji -- continuous integration, \glstext{CI} -- je jedna z rozšířených praktik extrémního programování a zapadá i do smýšlení DevOps a \glstext{SRE}~\cite{beyer2016site}. Ačkoliv to není nezbytně nutné~\cite{shore-ci}, proces začleňování změn je zpravidla podporován nějakým integračním serverem~\cite{fowler-ci}. Tato práce je se věnuje důkladnému porovnání těchto \CICD nástrojů: \textit{GitLab, Jenkins, GoCD, Drone, Concourse, CircleCI, Travis~CI} a \textit{Semaphore~CI}.

\subsection{Struktura práce}
    Práce je rozdělena na tři kapitoly. První z nich se věnuje definici a upřesnění termínů používaných ve zbytku práce.

    V druhé kapitole je definována metodika výběru porovnávaných \CICD nástrojů a v podkapitole je každý systém zhodnocen. V závěru této druhé kapitoly je shrnutí výhod a nevýhod otestovaných nástrojů. Podkapitoly o jednotlivých systémech jsou seřazeny a v pozdějších částech se odkazuji na již otestované nástroje. Přesto by každá část měla být čitelná sama o sobě; při výběru vhodného \CICD dokonce doporučuji nejprve nahlédnout na závěrečné shrnutí (strana~\pageref{overview}), vybrat systémy odpovídající Vašemu použití a následně číst příslušné podkapitoly.

    V poslední třetí kapitole je zdůvodněn výběr nástroje \textit{GitLab} jako nejvhodnějšího pro firmu \textit{manGoweb} a je zdokumentováno nasazení tohoto systému.
