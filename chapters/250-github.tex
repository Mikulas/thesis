\section{GitHub Actions}
    GitHub je \glstext{SaaS} a největší vývojářská platforma \cite{github-about}. Překvapivě od svého založení v roce 2008 neměl zabudované žádné \CI. Podporoval napojení na externí služby přes webhooks a až v roce 2012 přidali tzv.~\textit{commit status \glstext{API}}, které umožnilo \CI zapsat výsledek pipeline zpět do GitHubu. V roce 2018 GitHub toto přejmenoval na \textit{check runs \glstext{API}}, ale funkcionalita zůstala podobná.

    V roce byla 2019 zveřejněna úplně nová funkcionalita: GitHub Actions. Jde o obecný systém, který reaguje na různé GitHub události (nový commit, změna issue, deploy aplikace, \ldots) a spouští Docker kontejnery. Nepodporuje žádné složitější koncepty jako jsou služby na pozadí (například databáze jako závislost pro aplikační testy).

    V době psaní práce (první čtvrtletí 2019) jsou Actions v beta verzi a zuřivě se vyvíjí. Nebudu zkoumat konkrétní detaily, ale zaměřím se hlavně na vysokoúrovňový pohled a jak Actions zapadají do stávajícího ekosystému.

    Actions mohly úplně nahradit služby třetích stran jako jsou Travis, CircleCI, Semaphore a další. Spíš se ale dá očekávat, že se budou Actions s dalšími \CI nějakým způsobem kombinovat. Testy jednoúčelové a aplikovatelné na libovolný repozitář jsou pro Actions vhodné. Například statické otestování bash skriptů pomocí shellcheck \cite{ga-shellcheck}. Naopak aplikační testy na byznys logiku bude jednodušší spravovat v komplexnějším \CI.

    \subsection{Architektura GitHub Actions, možnosti konfigurace}
        Workflow se konfigurují souborem \code{.github/main.workflow} ve formátu \glstext{HCL}. GitHub také nabízí grafický editor, který konfigurační soubor generuje a ukládá do repozitáře. Samotná konfigurace obsahuje právě jeden blok \code{workflow}, ve kterém se definuje \glstext{API} událost která má workflow spustit. Zatím není možné spustit pipeline v reakci na víc než jednu událost.

        Zbytek workflow je acyklický graf docker kontejnerů (\code{action}) ke spuštění. Vývojář může změnit výchozí entrypoint a argumenty kontejneru, a definuje, na jaké jiné actions musí počkat. Kontejnery si mohou jednorázově předat informace souborem na disku. \todo{[QUICK] paralelni joby sdili nebo jak?}

        Kromě veřejných Docker obrazů dokáže GitHub kompilovat a spouštět i obrazy definované souborem Dockerfile v repozitáři. Navíc skvěle využívá koncept Docker štítků (\textit{labels}) pro definici metadat. Není tak potřeba jeden registr a Actions jsou úplně decentralizované \cite{ga-labels}. To dokonce umožňuje reimplementovat proprietární tenkou Workflow vrstvu, jako například vznikla v \code{nektos/act} \cite{nektos-act}.

    \subsection{Zabezpečení}
        GitHub jako \glstext{SaaS} přejímá skoro všechnu zodpovědnost za bezpečnost. Actions umožňují volat libovolné kontejnery a na vývojáři tak zůstává jenom kontrola Docker obrazů. V nejhorším případě kompromitovaný Docker obraz může při spuštění v pipeline ukrást aktuální stav repozitáře a může číst všechna definovaná tajemství, což budou typicky tokeny do služeb třetích stran.

    \subsection{Dostupnost}
        GitHub nedefinuje žádné \glstext{SLA} pro většinu placených plánů. Pro nejdražší korporátní plán nabízí \glstext{SLA} $99,95$ \%.

    \subsection{Praktické nasazení projektů}
        GitHub Actions jsou v neveřejné beta verzi a neměl jsem možnost vytvořit pipeline pro ukázkové projekty. \todo{[QUICK] kdyžtak doplnit github actions pro ukázkové projekty}
