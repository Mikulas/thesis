\section{GitHub Actions}
    GitHub, největší vývojářská platforma \cite{github-about}, neměl od svého založení v roce 2008 žádné zabudované \CI. Podporoval napojení na externí služby přes webhooks a v roce 2012 přidali tzv.~\textit{commit status \glstext{API}}, které umožnilo \CI zapsat výsledek pipeline zpět do GitHubu. V roce 2018 GitHub funkci přejmenoval na \textit{check runs \glstext{API}}, ale funkčně zůstala podobná.

    V roce byla 2019 zveřejněna úplně nová funkcionalita: GitHub Actions. Jde o obecný systém, který reaguje na různé GitHub události (nový commit, změna issue, deploy aplikace, \ldots) a spouští Docker kontejnery. Nepodporuje žádné složitější koncepty jako jsou služby na pozadí (například databáze jako závislost pro aplikační testy).

    Actions mohly úplně nahradit služby třetích stran jako jsou Travis, CircleCI, Semaphore a další. Spíš se ale dá očekávat, že se budou Actions s dalšími \CI nějakým způsobem kombinovat. Testy jednoúčelové a aplikovatelné na libovolný repozitář jsou pro Actions vhodné. Například statické otestování bash skriptů pomocí shellcheck \cite{ga-shellcheck}. Naopak aplikační testy na byznys logiku bude jednodušší spravovat v komplexnějším \CI.

    \subsection{Architektura GitHub Actions, možnosti konfigurace}
        \todo{Jaké má GitHub Actions části, jak se to deployuje, externí závislosti\ldots}\blind[2]

    \subsection{Rozšiřitelnost}
        \todo{Má GitHub Actions nějaké pluginy, dá se pro to scriptovat, jak je to bezpečné a jednoduché?}\blind[1]

    \subsection{Zabezpečení}
        \todo{Jaké jsou historická CVE? Jaká je izolace klientů? Co aplikace potřebuje za přístupy?}\blind[1]

    \subsection{Dostupnost}
        \todo{Může GitHub Actions běžet ve víc replikách? Jak se dělá upgrade? Jak stabilní to je?}\blind[1]

    \subsection{Integrace}
        \todo{Integrace GitHub Actions, oznámení na GitHub/GitLab/Bitbucket/\ldots}\blind[1]
        \todo{Možnosti deploy z GitHub Actions do cílového systému; k8s, sftp, openstack, \ldots}\blind[1]

    \subsection{Praktické nasazení projektů}
        \subsubsection{Projekt 1}
            \todo{Popsat deploy projektu 1 z GitHub Actions}\blind[1]

        \subsubsection{Projekt 2}
            \todo{Popsat deploy projektu 2 z GitHub Actions}\blind[1]

        \subsubsection{Projekt 3}
            \todo{Popsat deploy projektu 3 z GitHub Actions}\blind[1]

