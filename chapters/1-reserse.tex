\chapter{Rešerše}
    \todo{Obecne o resersi, proc jsem se zameril na nasledujici sekce a kam to míří.}
    \blind[4]

    \section{cloud}
        \todo{Definovat cloud a rozdíl oproti inhouse, proc je to vhodne pro CI/CD}
        \blind[2]

    \section{dostupnost}
        \todo{Uživatelská, administrační, vazba na zalohy a upgrade}
        \blind[3]

    \section{testování}
        \todo{Definovat testování v cloud}
        \todo{Popsat, jak lze snadno integrovat další služby jako například Messenger}
        \blind[4]

    \section{deploy}
        \todo{jake casti se nasazuji, jak a proc se to dela, dostupnost pri deploy}
        \blind[6]

    \section{definice CI/CD}
        \subsection{Continuous integration}
            Nejstarší zmínka o CI je projekt \textit{Infuse} z roku 1989 \cite{kaiser-infuse}. Jde o návrh systému testování komplexních modulárních projektů, který podle hierarchie spouští postupně jednotkové, integrační a akceptační testy.

            Fowler v článku z roku 2000 \cite{fowler-ci-original} definuje CI jako praktiku, při které vývojáři změny začleňují do sdíleného kódu jednou denně nebo častěji. Logická funkčnost (ale ne nutně úplnost byznysových funkcní) je pak po každé změně otestována automatickými testy. Pokud testy selžou, vývojář jehož změny zapříčinily při testování problémy je informován a očekává se, že chyby opraví.

            Na continuous integration navazuje celá řada praktik: feature branches \todo{seznam}.

            V aktualizovaném článku z roku 2006 Fowler \cite{fowler-ci} zmiňuje užitečnost CI serveru jako volitelnou -- ale praktickou -- podporu této praktiky.

        \subsection{Continuous delivery}
            \todo{define}
            Chad Wathington dále definuje Continuous integration jako podmnožinu CD \cite{fowler-go}.

        \subsection{Continuous deployment}


        \todo{Definovat ci/cd, popsat k cemu je to dobre atd.}
        \blind[6]

    \section{kontejnerizase, docker}
        \todo{Popsat co to je a proc je to vhodne na testovani a deploy}
        \blind[2]
