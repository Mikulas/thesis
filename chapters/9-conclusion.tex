\label{Conclusion}
    %\section{Comparison of solutions}
    %    \todo{}
        %\todo{Speed, cost...}
        %\todo{The FINAL comparison AFTER prototypes were built}
        %\todo{Table can be included}
        
    %\section{Work Done}
        %\todo{What is it I've done here, what consequences should it have}
        %\todo{I have built...  I have set the groundwork for...  }
        In this thesis, I was tasked with a problem and an idea for a solution.  Due to the fact that DVD drives have stopped being common in computers, giving out DVDs for people to try out the Fedora operating system was no longer a viable option.  Because mass producing USB flash drives is significantly more expensive, an alternative was conceived in a device that would write to flash drives on demand dubbed the Fedorator.
        
        The purpose of the device isn't just to write flash drives—that could be easily achieved with just a computer, or perhaps a commerical piece of hardware.  The Fedorator is intended to be present on marketing stands and draw attention and interest.
        
        I have first specified how this device would behave and what parameters should its look fulfill.  Next, I researched the options for the device's components—the hardware as well as the case.  I analyzed these options with regard to cost, ease of assembly, extensibility, and long term maintainability.  Based on this analysis, I created three distinct concepts.  Due to positive feedback, I went ahead and attempted to construct two of them.
        
        I turned a concept of a Fedorator driven by a Raspberry Pi into a reality.  I decided on the components, wrote the software and designed a case.  Then I showed the solution to be reasonable by building a functional prototype.  Constructing the prototype has allowed me to determine which things to improve on in the next iteration.
        
        I also strived to design and build a smaller variant on the Fedorator device.  I have done the research and performed some trials, but failed to secure a solution that would not require a large time investment.  This would be a good topic for a follow-up master's thesis. %  \todo{Yeah?}
        %\todo {could be a master's thesis topic}
        
        
    \section{Future Work}
        %\todo{How can we build on top of what I've made}
        The Fedorator prototype that was built is ready to be showcased at local events in order to garner feedback, and there are a few noted enhancements that can be made.
        
        The source code for all the work I have done is available on GitHub.  Thanks to the nature of open source, this means anybody can contribute improvements.  The Fedorator project has a future at the hands of me and other enthusiasts.
        
        %\todo{Thanks to being open-source and GitHub...}
        %\todo{"the prototype that I built is ready to be showcased at local events"}
        
