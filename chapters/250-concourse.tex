\section{Concourse}
    Concourse je minimalistické \CI. Zjednodušeně řečeno obsahuje pouze tři základní koncepty: pipeline (projekt), job a přihlášení.

    \todo{Concourse záměrně nemá žádné vazby na nic a jediné co podporuje je polling, takže to nefunguje pro spoustu use-cases.}

    \todo{popsat že má skvělou dokumentaci a popisují očekávaný disk, cpu, memory usage i high availability}

    \subsection{Architektura Concourse, možnosti konfigurace}
        Concourse využívá -- podobně jako Jenkins a GitLab -- architekturu kontrolerů (které nazývají \textit{web node}) a pracovních uzlů (\textit{worker node}). Na rozdíl od ostatních systémů ale má všechny části bezstavové. Data se persistují do externí PostgreSQL databáze. Díky tomu lze řídící rovinu libovolně škálovat a při provozu ve víc replikách má výbornou dostupnost.

        \missingfigure[figwidth=\columnwidth,figheight=7cm]{Architektura Concourse \label{pic:concourse-architecture}}
        \todo{\url{https://concourse-ci.org/concepts.html\#component-atc}}

        Nasazení Concourse je skvěle zdokumentované. Podobně jako u dalších distribuovaných systémů je potřeba před spuštěním jednotlivých komponent vygenerovat celou řadu klíčů. Na podepisování uživatelských session a interní komunikaci, identifikátor pro \glstext{SSH} server a pro každý \textit{worker node} jeden klíč k registraci ke kontroleru. Další nastavení je volitelné \code{concourse-cluster}.

        Pro rychlé otestování funkčnosti a drobné projekty nabízí Concourse zjednodušenou variantu spuštění. Jedním příkazem (\code{concourse quickstart}) se spustí kontroler a worker a automaticky se pro ně vygenerují a zaregistrují potřebné klíče.

        Concourse neřeší balancování Web instancí. Je možné -- a dokonce doporučené -- mít víc než jednu repliku kontroleru, ale je na administrátorovi aby příchozí požadavky nějakým způsobem směroval, Concourse pouze na každé Web instanci vystavuje \glstext{HTTP} \code{concourse-cluster}. V Kubernetes, OpenShift nebo podobném orchestrátoru můžeme přímo využít vestavěné služby, pro jiné systémy se ale nasazení Concourse komplikuje přidáním dalšího software pro load balancing.

        Worker instance musí být spuštěné pod root uživatelem. Podle firmy DigitalOcean je to proto, že worker potřebuje spravovat kontejnery \code{concourse-do}. To může v základním nastavení Dockeru ale libovolný linuxový uživatel v \code{docker} skupině. \code{docker-postinstall}. I s přístupem k Dockeru ale Concourse worker odmítal nastartovat a stále vyžadovat root uživatele.

        \todo{Popsat jak se zakládají pipelines a že je to dost nepřehledné}
        \todo{že při vytvoření pipeline to bylo pauznuté a nešlo to z UI vyřešit a musel jsem hledat command}
        \blind[2]

    \subsection{Rozšiřitelnost}
        Oproti Jenkins, kde se instalují pluginy, přistupuje Concourse k rozšiřitelnosti obráceně. Poskytuje několik základních zdrojů (pipeline, job) a \glstext{API}. Uživatelé můžou v rámci spuštěných kontejnerů volat cokoliv potřebují. Alternativní rozhraní a podobně je možné implementovat jako externí aplikaci. Concourse se snaží dělat jednu věc a dělat jí dobře.

        Concourse lze částečně rozšířit přes koncept \textit{resources} (zdroje). Ty se balíčkují jako kontejnery a jediný požadavek na ně je, že musí obsahovat programy \code{check}, \code{in} a \code{out} v cestě \code{/opt/resource} \cite{concourse-resource}. Podle konfigurace konkrétní pipeline se pak periodicky volá \code{check}, který má na za úkol vrátit seznam všech nových verzí. Volání \code{in} a \code{out} má za úkol dostat nějaké informace do Concourse pipeline, resp. je nahrát z pipeline ven. Příklad zdroje může být \glstext{RSS}, kde lze periodicky sledovat feed nějaké závislosti a při vydání nové verze automaticky spustit pipeline. Jiný příklad je zdroj git, který může nejenom reagovat na nové verze, ale i udělat nějaké změny v repozitáři. Resource se může napojit na libovolné \glstext{API} a existují stovky opensource integrací na všechno od monitorovacích nástrojů po \CD systémy jako je Kubernetes/Helm a řada dalších \cite{concourse-resource-list}.

        Protože některé \code{check} operace můžou být drahé a případně pomalé, přišel Concourse s možností definovat pro zdroje také token, kterým lze programově vynutit okamžité spuštění \code{check} z \glstext{API}. To je něco co by používala organizace pracující aktivně na několika málo z hodně repozitářů, kde je žádoucí spustit pipeline do pár vteřin od změny, ale není praktické s takovou periodou kontrolovat desítky repozitářů.

    \subsection{Zabezpečení}
        \todo{Každý v týmu může dělat cokoliv? Ověřit}\blind[1]
        \todo{Jaké jsou historická CVE? Jaká je izolace klientů? Co aplikace potřebuje za přístupy?}\blind[2]

    \subsection{Dostupnost}
        \todo{Může Concourse běžet ve víc replikách? Jak se dělá upgrade? Jak stabilní to je?}\blind[3]

    \subsection{Integrace}
        \todo{má to integrace na spoustu oauth (a obecně auth) viz concourse quickstart --help}\blind[1]
        \todo{\url{https://medium.com/concourse-ci/designing-a-dashboard-for-concourse-fe2e03248751}}
        \todo{Integrace Concourse, oznámení na GitHub/GitLab/Bitbucket/\ldots}\blind[1]
        \todo{Možnosti deploy z Concourse do cílového systému; k8s, sftp, openstack, \ldots}\blind[1]

    \subsection{Praktické nasazení projektů}
        Navzdory skvělé dokumentaci pro nasazení Concourse clusteru bylo pro mě vytváření pipeline a jejich správa utrpení.

        \subsubsection{Projekt 1}
            \todo{Popsat deploy projektu 1 z Concourse}\blind[2]

        \subsubsection{Projekt 2}
            \todo{Popsat deploy projektu 2 z Concourse}\blind[2]

        \subsubsection{Projekt 3}
            \todo{Popsat deploy projektu 3 z Concourse}\blind[2]
