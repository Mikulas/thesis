\section{GoCD}
    Systém GoCD vychází z projektu Cruise \cite{thoughtworks-gocd}. Oba projekty vznikly ve firmě ThoughtWorks, kde pracoval průkopník a zastánce praktik \textit{extrémního programování} M. Fowler \cite{fowler-go}. Fowler systém Cruise doporučoval ve známém článku o \CI \cite{fowler-ci}.

    \subsection{Architektura GoCD, možnosti konfigurace}
        \todo{Jaké má GoCD části, jak se to deployuje, externí závislosti\ldots}\blind[4]

    \subsection{Rozšiřitelnost}
        \todo{Má GoCD nějaké pluginy, dá se pro to scriptovat, jak je to bezpečné a jednoduché?}\blind[3]

    \subsection{Zabezpečení}
        \todo{Jaké jsou historická CVE? Jaká je izolace klientů? Co aplikace potřebuje za přístupy?}\blind[3]

    \subsection{Dostupnost}
        \todo{Může GoCD běžet ve víc replikách? Jak se dělá upgrade? Jak stabilní to je?}\blind[3]

    \subsection{Integrace}
        \todo{Integrace GoCD, oznámení na GitHub/GitLab/Bitbucket/\ldots}\blind[2]
        \todo{Možnosti deploy z GoCD do cílového systému; k8s, sftp, openstack, \ldots}\blind[5]

    \subsection{Praktické nasazení projektů}
        \subsubsection{Projekt 1}
            \todo{Popsat deploy projektu 1 z GoCD}\blind[2]

        \subsubsection{Projekt 2}
            \todo{Popsat deploy projektu 2 z GoCD}\blind[2]

        \subsubsection{Projekt 3}
            \todo{Popsat deploy projektu 3 z GoCD}\blind[2]
