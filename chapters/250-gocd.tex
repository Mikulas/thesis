\section{GoCD}
    Systém GoCD vychází z projektu Cruise \cite{thoughtworks-gocd}. Oba projekty vznikly ve firmě ThoughtWorks, kde pracoval průkopník a zastánce praktik \textit{extrémního programování} M. Fowler \cite{fowler-go}. Fowler systém Cruise doporučoval ve známém článku o \CI \cite{fowler-ci}.

    \subsection{Architektura GoCD, možnosti konfigurace}
        GoCD stejně jako většina ostatních \CI staví na architektuře jednoho kontrolního serveru a řadě interních nebo externích agentů, které se starají o spuštění jednotlivých jobů. Ve výchozím nastavení jsou data persistována v embedované H2 databázi na server procesu. Instalace GoCD je díky tomu jednoduchá, protože stačí zaregistrovat externí repozitář a nainstalovat balíček pro server a pro agenty.

        Po prvním zobrazení webového rozhraní serveru se zobrazuje quick-start, který provádí vytvořením nové pipeline. Přestože GoCD podporuje \textit{Pipelines as code}, očekávaný primární vstup je klikání v administraci \cite{gocd-pas}. Načítání externích souborů je vyřešeno rozšířením, jehož použití se definuje v hlavní \glstext{XML} konfiguraci na webu. Separovaná konfigurace tak není kompletní a závisí na další ruční konfiguraci.

    \subsection{Rozšiřitelnost}
        GoCD nabízí několik možností, jak rozšiřovat výchozí funkcionalitu \cite{gocd-extensions}.

\cite{gocd-plugins}
        \todo{Má GoCD nějaké pluginy, dá se pro to scriptovat, jak je to bezpečné a jednoduché?}\blind[3]

    \subsection{Zabezpečení}
        v zakladu admin access bez zabezpeceni
        \todo{Jaké jsou historická CVE? Jaká je izolace klientů? Co aplikace potřebuje za přístupy?}\blind[3]

    \subsection{Dostupnost}
        GoCD agenti jsou z principu stavové aplikace a stejně jako u všech ostatních \CI jejich výpadek přijdeme o spuštěné joby. Může jich ale běžet mnoho a pomocí rolling update je lze aktualizovat bez výpadku.

        Server je bohužel také stavový a může běžet pouze v jedné replice. GoCD prodává velmi drahý \textit{Business Continuity Addon}, který dokáže udržovat standby repliku \cite{gocd-ha}. Failover proces ale není nijak automatizovaný a povýšení na primární repliku vyžaduje restart GoCD serveru. U GoCD nelze udělat perfektní HA bez ztráty žádnéoho requestu když vypadne primární replika.

        \todo{Jak se dělá upgrade? Jak stabilní to je?}\blind[1]

    \subsection{Integrace}
        \todo{Integrace GoCD, oznámení na GitHub/GitLab/Bitbucket/\ldots}\blind[2]
        \todo{Možnosti deploy z GoCD do cílového systému; k8s, sftp, openstack, \ldots}\blind[5]

    \subsection{Praktické nasazení projektů}
        \subsubsection{Projekt 1}
            \todo{ze jsem bojoval s env, jinak vlastne v pohode, prirovanat k jenkins}
            \todo{Popsat deploy projektu 1 z GoCD}\blind[2]

        \subsubsection{Projekt 2}
            \todo{Popsat deploy projektu 2 z GoCD}\blind[2]

        \subsubsection{Projekt 3}
            \todo{Popsat deploy projektu 3 z GoCD}\blind[2]
