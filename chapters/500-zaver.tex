\label{Závěr}
    Náplní mé diplomové práce byl průzkum veřejně dostupných řešení pro \CICD. Porovnával jsem GitLab, Jenkins, Concourse, Drone, GoCD, CircleCI, Semaphore CI a Travis CI. Pro všechny jsem kódem vytvořil virtuální stroj a aplikaci jsem nainstaloval a nakonfiguroval nástrojem Chef. Pro každý z~těchto osmi systémů jsem vytvořil tři různé specifikace \CI: pro statický web, pro aplikaci s~komplexními závislosti a kontejnerizovanou aplikaci. Tím jsem prakticky vyzkoušel možnosti daných systémů. Předem stanovenou metodikou jsem jednotlivé systémy ohodnotil. Díky velké rozmanitosti nástrojů nelze žádný vyzdvihnout nad všechny ostatní.

    GitLab je veřejně dostupné self-hosted \CI s~nejvíc funkcemi, důrazem na bezpečnost a kvalitním uživatelským rozhraním. Drone je minimalistický nástroj, vhodný pokud už máme zavedený nějaký systém na správu repozitářů a úkolů. Jenkins je přes svoji popularitu ve všech kategoriích zastíněn jinými \CICD.

    Nasadil jsem systém pro podporu \CICD ve firmě manGoweb,~s.r.o. Sepsal jsem požadavky a podle nich vybral GitLab jako nejvhodnější a ekonomicky nejsmysluplnější řešení. Vytvořil jsem automatizaci, díky které firma může nasazovat aplikace z~\CI systému do beta a produkčních Kubernetes clusterů.

    V~průběhu psaní práce vznikla řada zajímavých \CICD systémů, které by bylo vhodné prozkoumat. Jedním z~nich je například projekt Tekton Pipelines, který využívá primitiv z~Kubernetes. Po oficiálním vydání a stabilizaci bude jistě také zajímavé znovu ohodnotit GitHub Actions, které jsou v~době psaní ve veřejné beta verzi.
